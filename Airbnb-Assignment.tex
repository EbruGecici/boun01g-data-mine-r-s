% Options for packages loaded elsewhere
\PassOptionsToPackage{unicode}{hyperref}
\PassOptionsToPackage{hyphens}{url}
%
\documentclass[
]{article}
\usepackage{lmodern}
\usepackage{amssymb,amsmath}
\usepackage{ifxetex,ifluatex}
\ifnum 0\ifxetex 1\fi\ifluatex 1\fi=0 % if pdftex
  \usepackage[T1]{fontenc}
  \usepackage[utf8]{inputenc}
  \usepackage{textcomp} % provide euro and other symbols
\else % if luatex or xetex
  \usepackage{unicode-math}
  \defaultfontfeatures{Scale=MatchLowercase}
  \defaultfontfeatures[\rmfamily]{Ligatures=TeX,Scale=1}
\fi
% Use upquote if available, for straight quotes in verbatim environments
\IfFileExists{upquote.sty}{\usepackage{upquote}}{}
\IfFileExists{microtype.sty}{% use microtype if available
  \usepackage[]{microtype}
  \UseMicrotypeSet[protrusion]{basicmath} % disable protrusion for tt fonts
}{}
\makeatletter
\@ifundefined{KOMAClassName}{% if non-KOMA class
  \IfFileExists{parskip.sty}{%
    \usepackage{parskip}
  }{% else
    \setlength{\parindent}{0pt}
    \setlength{\parskip}{6pt plus 2pt minus 1pt}}
}{% if KOMA class
  \KOMAoptions{parskip=half}}
\makeatother
\usepackage{xcolor}
\IfFileExists{xurl.sty}{\usepackage{xurl}}{} % add URL line breaks if available
\IfFileExists{bookmark.sty}{\usepackage{bookmark}}{\usepackage{hyperref}}
\hypersetup{
  pdftitle={NYC Airbnb Data Assignment},
  pdfauthor={Data Mine'R's},
  hidelinks,
  pdfcreator={LaTeX via pandoc}}
\urlstyle{same} % disable monospaced font for URLs
\usepackage[margin=1in]{geometry}
\usepackage{graphicx}
\makeatletter
\def\maxwidth{\ifdim\Gin@nat@width>\linewidth\linewidth\else\Gin@nat@width\fi}
\def\maxheight{\ifdim\Gin@nat@height>\textheight\textheight\else\Gin@nat@height\fi}
\makeatother
% Scale images if necessary, so that they will not overflow the page
% margins by default, and it is still possible to overwrite the defaults
% using explicit options in \includegraphics[width, height, ...]{}
\setkeys{Gin}{width=\maxwidth,height=\maxheight,keepaspectratio}
% Set default figure placement to htbp
\makeatletter
\def\fps@figure{htbp}
\makeatother
\setlength{\emergencystretch}{3em} % prevent overfull lines
\providecommand{\tightlist}{%
  \setlength{\itemsep}{0pt}\setlength{\parskip}{0pt}}
\setcounter{secnumdepth}{-\maxdimen} % remove section numbering
\ifluatex
  \usepackage{selnolig}  % disable illegal ligatures
\fi

\title{NYC Airbnb Data Assignment}
\author{Data Mine'R's}
\date{8/26/2020}

\begin{document}
\maketitle

{
\setcounter{tocdepth}{3}
\tableofcontents
}
\hypertarget{introduction}{%
\subsection{1. Introduction}\label{introduction}}

\hypertarget{what-is-airbnb}{%
\subsubsection{1.1. What is Airbnb?}\label{what-is-airbnb}}

Airbnb is an online marketplace since 2008, which connects people who
want to rent their homes with people who are looking for accommodations
in a particular location. It covers more than 81,000 cities and 191
countries worldwide. The company ,which is based in San Francisco,
Califarnia, does not own any of the property listings, but it receives
commissions from each booking like a broker. The name ``Airbnb'' comes
from ``air mattress Bed and Breakfast.'' The Airbnb logo is called the
Bélo, which is a short version for saying `Belong Anywhere'. Airbnb
hosts list many different kinds of properties such as private rooms,
apartments, shared rooms, houseboats, entire houses, etc.

\hypertarget{airbnb-dataset}{%
\subsubsection{1.2. Airbnb Dataset}\label{airbnb-dataset}}

This dataset describes the listing activity and metrics in NYC for 2019.
It includes all the necessary information in order to find out more
about hosts, prices, geographical availability, and necessary
information to make predictions and draw conclusions for NYC. The
explanation of the variables in our data, which consists of 16 columns
and 48,895 rows, will be made in the next part. The data used in this
assignment is called \textbf{New York City Airbnb Open Data} which is
downloaded from
\href{https://www.kaggle.com/dgomonov/new-york-city-airbnb-open-data}{Kaggle}.
This public dataset is a part of Airbnb, and the original source can be
found on this \href{http://insideairbnb.com/}{website}.

\hypertarget{objectives}{%
\subsubsection{1.3. Objectives}\label{objectives}}

In this assignment, we will perform an exploratory data analysis(EDA) in
order to investigate each of the variables and also come up with a
conclusion for the relationship between variables. The main purpose is
to identify which variables affect the price mostly, and include a
regression model with price as a response variable. In addition to
these, we will explore which neighborhood groups and room types are the
most popular ones among the guests, and which hosts are the most
preferred ones. The processes during the assignment can be listes as
below:

\begin{enumerate}
\def\labelenumi{\arabic{enumi}.}
\tightlist
\item
  Data Preprocessing
\item
  Data Manipulation
\item
  Data Visualization
\item
  Interactive Shiny App
\item
  Forecasting
\end{enumerate}

\hypertarget{data-explanation}{%
\subsection{2. Data Explanation}\label{data-explanation}}

\hypertarget{variables}{%
\subsubsection{2.1. Variables}\label{variables}}

This dataset contains 16 features about Airbnb listings within New York
City. Below are the features with their descriptions:

\begin{enumerate}
\def\labelenumi{\arabic{enumi}.}
\tightlist
\item
  \texttt{id}: Listing ID (numeric variable)
\item
  \texttt{name}: Listing Title (categorical variable)
\item
  \texttt{host\_id}: ID of Host (numeric variable)
\item
  \texttt{host\_name}: Name of Host (categorical Variable)
\item
  \texttt{neighbourhood\_group}: Neighbourhood group that contains
  listing (categorical variable)
\item
  \texttt{neighbourhood}: Neighbourhood group that contains listing
  (categorical variable)
\item
  \texttt{latitude}: Latitude of listing (numeric variable)
\item
  \texttt{longitude}: Longitude of listing (numeric variable)
\item
  \texttt{room\_type}: Type of the offered property (categorical
  variable)
\item
  \texttt{price}: Price per night in USD (numeric variable)
\item
  \texttt{minimum\_nights}: Minimum number of nights required to book
  listing (numeric variable)
\item
  \texttt{number\_of\_reviews}: Total number of reviews that listing has
  (numeric variable)
\item
  \texttt{last\_review}: Last rent date of the listing (date variable)
\item
  \texttt{reviews\_per\_month}: Total number of reviews divided by the
  number of months that the listing is active (numeric variable)
\item
  \texttt{calculated\_host\_listings\_count}: Amount of listing per host
  (numeric variable)
\item
  \texttt{availability\_365}: Number of days per year the listing is
  active (numeric variable)
\end{enumerate}

\hypertarget{used-libraries}{%
\subsubsection{2.2. Used Libraries}\label{used-libraries}}

We have used several packages during the analysis of the historical data
of Airbnb in NYC in order to make data manipulation and visualization.
The list of packages used in this assignment can be seen below:

\begin{enumerate}
\def\labelenumi{\arabic{enumi}.}
\tightlist
\item
  tidyverse
\item
  lubridate
\item
  tinytex
\item
  wordcloud
\item
  shiny
\item
  knitr
\item
  data.table
\end{enumerate}

\textbf{Buraya kullanacağımız diğer paketleri de ekleriz. Library komutu
ile paketleri yüklediğimiz ve datayı import ettiğimiz kodu koyarız.
glimpse() ile dataya bakarız}

\hypertarget{dublicate-and-missing-data}{%
\subsubsection{2.3. Dublicate and Missing
Data}\label{dublicate-and-missing-data}}

\textbf{Dublicate ve NA var mı diye kontrol ettiğimiz kodu ekleriz. }

\hypertarget{summary-of-data}{%
\subsubsection{2.4. Summary of Data}\label{summary-of-data}}

\textbf{summary ile dataya baktıgımız kodu koyarız.}

\hypertarget{analysis}{%
\subsection{3. Analysis}\label{analysis}}

\end{document}
